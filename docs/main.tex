\documentclass[12pt]{report}
\usepackage[T1]{fontenc} % Obsługa polskich znaków
\usepackage[utf8]{inputenc}
\usepackage[polish]{babel}
\usepackage{geometry}
\geometry{a4paper, margin=1in}
\usepackage{tocbibind} % Dodaje spis treści do spisu treści
\usepackage{hyperref} % Pakiet do obsługi linków

\title{Narzędzie wspierające tworzenie testów dla aplikacji webowych}
\author{
    Łukasz Kowalewski \\ % Autor pracy
    \vspace{0.5cm} % Odstęp między autorem a promotorem
    Promotor: dr inż. Marcin Adamski % Promotor
}
\date{}

\begin{document}

\maketitle

\tableofcontents
\newpage

\chapter{Wstęp}
W niniejszym wstępie postaram się uzasadnić istotność tematu pracy inżynierskiej. Tytuł pracy "Narzędzie wspierające tworzenie testów dla aplikacji webowych" opisuje funkcję, jaką stworzony w ramach pracy inżynierskiej program ma za zadanie spełnić. Jego rolą jest dostarczenie użytkownikowi narzędzia, które pozwoliłoby mu dostarczać oprogramowania oraz testów lepszej jakości w krótszym czasie. Następne rozdziały postarają się omówić problematykę zagadnienia oraz możliwe rozwiązania. By zrozumieć i tym samym znaleźć motywację, by owe zagadnienia analizować, należałoby uzasadnić potrzebę tworzenia takiego rodzaju oprogramowania.

Testy stanowią kluczowy element każdego zaawansowanego oraz dojrzałego systemu informatycznego bądź aplikacji. Jest to fakt uznany i łatwy do wykazania, zwracając uwagę choćby na istnienie specjalnych oddziałów w korporacjach zajmujących się wyłącznie tworzeniem testów, bądź zauważając wysoki popyt na programistów wyspecjalizowanych w tej dziedzinie. To właśnie takie korporacje najczęściej tworzą zaawansowane narzędzia typu omawianego w powyższej pracy inżynierskiej. W tak dużych projektach wszelkie błędy systemu potrafią odbić się na bardzo wysokich stratach finansowych, one właśnie stanowią główne źródło motywacji, by w ogóle te testy tworzyć. Jednocześnie warto zwrócić uwagę na wprost przeciwny obraz sytuacji dla projektów średnich bądź małych rozmiarów. W tego typu przedsięwzięciach często brakuje źródła motywacji m.in. w postaci potencjalnych wysokich strat finansowych, co skutkuje sytuacją, w której element testów jest pomijany ze względu na wysoki koszt czasowy ich tworzenia.

Głównym celem omawianej aplikacji będzie wyjście naprzeciw potrzebom ułatwienia tworzenia w krótkim czasie dużej ilości prostych testów tak, by małe i średnie projekty możliwie niskim kosztem uzyskały wysokie pokrycie testowe.

\section{Cel i zakres pracy}
Celem niniejszej pracy inżynierskiej jest opracowanie narzędzia, które znacząco usprawni proces tworzenia testów automatycznych w projektach webowych. Z tego względu praca skupia się nie tylko na samym wytworzeniu aplikacji, ale również na omówieniu metod i dobrych praktyk towarzyszących procesowi testowania. 

Tworzone oprogramowanie ma rozwiązać przede wszystkim następujące problemy:
\begin{itemize}
    \item Wysoki próg wejścia w automatyzację testów dla początkujących zespołów – narzędzie powinno dostarczyć przyjazne mechanizmy generowania przykładowych skryptów i scenariuszy testowych.
    \item Długi czas przygotowywania testów w małych i średnich projektach – planuje się zapewnić funkcje, które przyspieszą proces konfiguracji i pisania testów (np. wstępne generowanie kodu testowego).
    \item Integracja z istniejącymi technologiami – narzędzie będzie wykorzystywać znane biblioteki (takie jak Selenium lub Playwright) oraz umożliwiać łatwe włączenie do popularnych pipeline’ów CI/CD.
\end{itemize}

Końcowym rezultatem jest działające oprogramowanie, którego zastosowanie pozwoli na efektywne tworzenie testów automatycznych oraz ich integrację z cyklem wytwarzania oprogramowania. Dokumentacja pracy przybliży dodatkowo zarówno aspekty teoretyczne (przegląd frameworków, omawianie strategii testowania), jak i praktyczne (szczegółowe omówienie kodu, przykłady uruchomienia testów i wdrożenia w środowisku CI/CD).

Zakres opracowania obejmuje zatem:
\begin{itemize}
    \item Analizę dostępnych narzędzi i bibliotek testowych dla aplikacji webowych,
    \item Zaprojektowanie i implementację modularnego narzędzia generującego testy,
    \item Prezentację możliwych sposobów dalszego rozwoju aplikacji, w tym plan rozbudowy funkcjonalności i integracji z innymi platformami.
\end{itemize}

\section{Struktura pracy}
W dalszych rozdziałach zaprezentowano kluczowe elementy związane z projektowaniem i tworzeniem narzędzia do wspierania testów automatycznych:

\begin{itemize}
    \item \textbf{Rozdział 2} omawia przykłady zastosowań testów automatycznych, przedstawiając ich różnorodność (testy jednostkowe, integracyjne oraz end-to-end).
    \item \textbf{Rozdział 3} zawiera przegląd popularnych technologii i bibliotek wspierających testowanie aplikacji webowych (m.in. frameworki testowe, narzędzia typu record-and-play, rozwiązania CI/CD).
    \item \textbf{Rozdział 4} skupia się na różnych strategiach generowania testów, w tym podejściach opartych na rejestrowaniu akcji i modelach systemu.
    \item \textbf{Rozdział 5} przedstawia architekturę i interfejs tworzonej aplikacji, opisując poszczególne moduły i sposób ich komunikacji.
    \item \textbf{Rozdział 6} zawiera omówienie kluczowych elementów kodu źródłowego narzędzia oraz przybliża zastosowane biblioteki.
    \item \textbf{Rozdział 7} prezentuje praktyczne przykłady użycia aplikacji, zarówno w trybie interaktywnym, jak i zintegrowanym w potoku CI/CD.
    \item \textbf{Rozdział 8} stanowi podsumowanie pracy, w którym zawarto wnioski końcowe dotyczące realizacji założeń oraz wskazano możliwe kierunki rozwoju narzędzia.
\end{itemize}

\chapter{Przykłady zastosowań testów}
\label{chap:przyklady-testow}
Współczesne aplikacje webowe – bez względu na ich skalę – wymagają odpowiedniego poziomu kontroli jakości, a testy automatyczne są jednym z najsolidniejszych sposobów osiągnięcia tego celu. W niniejszym rozdziale przedstawiono typowe rodzaje testów stosowane przy rozwoju oprogramowania, ze szczególnym uwzględnieniem aplikacji internetowych. Rozważymy zarówno proste testy jednostkowe, jak i bardziej rozbudowane testy integracyjne czy kompleksowe testy end-to-end (E2E). Każdy z wymienionych poziomów testowania ma swoją specyfikę oraz obszar zastosowań – dzięki temu w bardziej złożonych projektach można zastosować podejście wielopoziomowe, co pozwala uniknąć błędów na różnych etapach.

Testy te pełnią także niebagatelną rolę w procesie ciągłej integracji i dostarczania (CI/CD). Automatyczne uruchamianie zestawów testowych przed wdrożeniem oprogramowania minimalizuje ryzyko wypuszczenia wadliwej wersji aplikacji. W dalszej części rozdziału pokazane zostaną najważniejsze cechy trzech kluczowych kategorii testów, a także krótki komentarz dotyczący korzyści i wyzwań związanych z ich stosowaniem.

\section{Testy jednostkowe}
\label{sec:testy-jednostkowe}
Testy jednostkowe (ang. \emph{unit tests}) skupiają się na najmniejszych fragmentach oprogramowania, zazwyczaj na pojedynczych funkcjach, metodach lub klasach. Celem takich testów jest sprawdzenie poprawności działania konkretnego komponentu w oderwaniu od reszty systemu. Dzięki temu programiści mogą szybko wykryć regresje w kodzie w momencie wprowadzania zmian lub nowych funkcjonalności.

\subsection*{Charakterystyka i zalety testów jednostkowych}
\begin{itemize}
    \item \textbf{Wczesne wykrywanie błędów:} niewielki zakres testowanych fragmentów kodu umożliwia szybką diagnozę przyczyn problemu.
    \item \textbf{Łatwość uruchamiania:} testy te są zwykle szybkie i mało zasobożerne, dzięki czemu mogą być wykonywane nawet przy każdej kompilacji.
    \item \textbf{Wspieranie refaktoryzacji:} jeśli są dobrze napisane, stanowią swego rodzaju „siatkę bezpieczeństwa” – zmiana kodu pociąga za sobą natychmiastową weryfikację, czy nadal działa on tak, jak powinien.
\end{itemize}

\subsection*{Miejsce w cyklu życia aplikacji}
Zazwyczaj testy jednostkowe tworzone są równolegle z implementacją funkcji bądź metod, często w metodyce \emph{Test-Driven Development} (TDD). Nawet jeśli programiści nie stosują formalnie TDD, to i tak przeważnie piszą testy tuż po implementacji kluczowych funkcji. Pozwala to na bieżąco weryfikować stabilność kodu i ogranicza ryzyko dalszych problemów w bardziej złożonych warstwach systemu.

\subsection*{Znaczenie dla jakości oprogramowania}
Choć testy jednostkowe nie gwarantują wykrycia wszystkich błędów (szczególnie tych na poziomie integracji czy funkcjonalności kompleksowej), stanowią one fundament rzetelnego procesu testowego. Dzięki nim łatwiej utrzymać wysoką jakość aplikacji w długofalowym rozwoju, ograniczając liczbę niespodziewanych błędów w kluczowych komponentach kodu.

\section{Testy integracyjne}
\label{sec:testy-integracyjne}
Testy integracyjne (ang. \emph{integration tests}) służą do weryfikacji współdziałania między różnymi komponentami lub modułami systemu. W odróżnieniu od testów jednostkowych, które koncentrują się na pojedynczych funkcjach, testy integracyjne sprawdzają poprawność przepływu danych oraz komunikacji na wyższym poziomie.

\subsection*{Główny cel i zakres}
Podstawowym celem testów integracyjnych jest upewnienie się, że wszystkie części aplikacji współpracują ze sobą w sposób zgodny z założeniami. W projektach webowych może to obejmować m.in. komunikację serwera z bazą danych, przepływ danych między mikrousługami, a także integrację z zewnętrznymi API. Zakres takiego testu jest zatem szerszy niż w przypadku testów jednostkowych, ponieważ obejmuje wiele komponentów, a niekiedy także usługi firm trzecich.

\subsection*{Przykłady zastosowań w aplikacjach webowych}
\begin{itemize}
    \item \textbf{Sprawdzenie warstwy API i bazy danych:} testy integracyjne mogą weryfikować, czy zapytania HTTP wysyłane przez front-end trafiają do warstwy serwerowej i są poprawnie przetwarzane, a także czy baza danych zwraca oczekiwane wyniki.
    \item \textbf{Integracja z usługami zewnętrznymi:} w przypadku aplikacji korzystających z serwisów płatności czy map, testy integracyjne pozwalają sprawdzić, czy dane usługi są właściwie wywoływane i czy zwracają poprawne odpowiedzi.
    \item \textbf{Weryfikacja logiki po stronie serwera:} testy mogą obejmować zaszyte w kodzie warunki, reguły biznesowe czy sekwencje operacji, które są wykonywane przy interakcji między kilkoma modułami.
\end{itemize}

\subsection*{Korzyści i wyzwania}
\textbf{Zaletą} testów integracyjnych jest to, że pozwalają wykryć problemy w miejscach, gdzie komponenty łączą się i współpracują. Dzięki temu można uniknąć sytuacji, w której pojedyncze moduły działają poprawnie samodzielnie, ale po zestawieniu razem pojawiają się nieoczekiwane błędy. \\
\textbf{Wyzwania} to z kolei konieczność skonfigurowania odpowiednich środowisk testowych i danych. Wymaga to często stawiania testowych baz danych lub stubów i mocków dla zewnętrznych API. Przy dużych projektach może być to stosunkowo pracochłonne i kosztowne w utrzymaniu. 

% Przykładowy schemat (opcjonalnie):
% \begin{figure}[h]
%     \centering
%     \fbox{\rule{0pt}{40pt}\rule{0.9\linewidth}{0pt}}
%     \caption{Schemat interakcji modułów w testach integracyjnych.}
%     \label{fig:integration-tests}
% \end{figure}

\section{Testy end-to-end (E2E)}
\label{sec:testy-end-to-end}
Testy end-to-end (ang. \emph{end-to-end tests}) to najbardziej rozbudowany rodzaj testów funkcjonalnych, w których symuluje się rzeczywiste zachowanie użytkownika końcowego (lub interakcję system–system) w całym przekroju aplikacji. Obejmują one wszystkie warstwy oprogramowania, od interfejsu użytkownika po bazę danych i usługi zewnętrzne.

\subsection*{Na czym polega koncepcja testów E2E w aplikacjach webowych}
Główną ideą testów E2E jest sprawdzenie, czy cała aplikacja działa poprawnie, gdy rozpatrujemy ją jako kompletny produkt. Taki test może obejmować np.:
\begin{enumerate}
    \item Uruchomienie przeglądarki i wejście na stronę logowania.
    \item Wprowadzenie danych uwierzytelniających i przejście do właściwej części aplikacji.
    \item Skorzystanie z określonych funkcji (np. wyszukiwanie produktu, dodawanie do koszyka, finalizacja zamówienia).
    \item Weryfikację rezultatów w bazie danych czy też w komunikatach wyświetlanych użytkownikowi.
\end{enumerate}
Cały przepływ jest obserwowany oraz weryfikowany pod kątem spełnienia wymagań biznesowych i oczekiwań użytkownika.

\subsection*{Kiedy i dlaczego warto je stosować}
\begin{itemize}
    \item \textbf{Pełne pokrycie scenariuszy biznesowych:} testy E2E sprawdzają logikę aplikacji od początku do końca, co pozwala mieć pewność, że kluczowe ścieżki są wolne od krytycznych błędów.
    \item \textbf{Symulacja realnych warunków:} testy odzwierciedlają działania prawdziwego użytkownika, dzięki czemu można wychwycić problemy wynikające z kolejności czynności czy specyfiki interfejsu.
    \item \textbf{Wysoka wiarygodność:} przejście całego procesu „na żywo” daje dobre potwierdzenie, że aplikacja działa w sposób oczekiwany.
\end{itemize}
Z drugiej strony testy E2E są zwykle \textbf{najwolniejsze} i \textbf{najtrudniejsze w utrzymaniu}, ponieważ wymagają uruchomienia wszystkich niezbędnych elementów systemu (np. front-end, back-end, baza danych, zewnętrzne usługi). Dlatego stosuje się je najczęściej do krytycznych ścieżek użytkownika oraz do ostatecznej weryfikacji przed wdrożeniem.

\subsection*{Przykładowe narzędzia}
Do najpopularniejszych rozwiązań wspierających testy end-to-end w aplikacjach webowych należą:
\begin{itemize}
    \item \textbf{Selenium WebDriver} – klasyczne narzędzie automatyzujące przeglądarkę, szeroko wspierane przez społeczność.
    \item \textbf{Cypress} – nowoczesny framework skupiający się na testach front-endu z szybkim sprzężeniem zwrotnym.
    \item \textbf{Playwright} – rozwijane przez Microsoft narzędzie oferujące bogate możliwości w zakresie automatyzacji przeglądarek (Chromium, Firefox, WebKit).
    \item \textbf{TestCafe} – stosunkowo proste rozwiązanie pozwalające pisać testy E2E w JavaScripcie i TypeScripcie, bez konieczności instalacji dodatkowych sterowników przeglądarki.
\end{itemize}
Wybór konkretnego narzędzia zależy od technologii użytych w danym projekcie, wymagań zespołu oraz preferencji co do języka programowania. 

\chapter{Dostępne technologie}
Rozdział trzeci poświęcony jest omówieniu najpopularniejszych narzędzi i bibliotek wspierających proces testowania aplikacji webowych. Wybór odpowiednich frameworków i rozwiązań znacząco wpływa na wydajność tworzenia testów, ich utrzymanie oraz integrację z pozostałymi elementami środowiska projektowego. W pierwszej sekcji przyjrzymy się najistotniejszym \textit{frameworkom testowym}, dzięki którym można w prosty sposób pisać i organizować testy. Następnie zostaną omówione narzędzia usprawniające \emph{generowanie} i \emph{rejestrowanie} testów, a także najważniejsze rozwiązania integrujące się z procesami \emph{CI/CD}.

Celem rozdziału jest wskazanie kluczowych czynników wyboru poszczególnych technologii oraz prezentacja ich mocnych i słabych stron w kontekście pracy inżynierskiej. Zrozumienie różnic między dostępnymi narzędziami pozwoli na świadomą decyzję przy projektowaniu i implementacji aplikacji wspierającej automatyczne testowanie, tak aby finalne rozwiązanie było dostosowane do różnych warunków projektowych.

\section{Frameworki testowe}
Framework testowy to zestaw narzędzi i bibliotek, które pomagają w tworzeniu, organizowaniu oraz uruchamianiu testów w sposób powtarzalny i zautomatyzowany. W przypadku testów aplikacji webowych, frameworki te mogą oferować szerokie wsparcie dla różnych języków programowania (np. JavaScript, Python, Java, C\#) i typów testów (jednostkowe, integracyjne, end-to-end).

\subsection*{Przegląd wybranych rozwiązań}
\begin{itemize}
    \item \textbf{JUnit / TestNG} (dla Javy) -- bardzo popularne narzędzia w świecie Javy. JUnit jest często stosowany do testów jednostkowych, podczas gdy TestNG oferuje szerokie możliwości konfiguracyjne i równoległe uruchamianie testów.
    \item \textbf{Pytest} (dla Pythona) -- charakteryzuje się prostą składnią, a także bogatym zestawem wtyczek (tzw. \emph{plugins}). Dzięki nim łatwo integrować testy z innymi narzędziami (np. do raportowania).
    \item \textbf{Mocha / Jest} (dla JavaScript / Node.js) -- Mocha jest lekkim frameworkiem, który pozwala tworzyć testy asynchroniczne i synchroniczne. Z kolei Jest (rozwijany przez Facebook) jest często wybierany do testowania aplikacji React czy w ogóle projektów front-endowych.
    \item \textbf{Cucumber} -- narzędzie pozwalające na pisanie testów w formie zrozumiałej dla nietechnicznych interesariuszy, wykorzystuje tzw. \emph{Gherkin syntax}. Istotne w projektach nastawionych na \emph{BDD} (\emph{Behavior-Driven Development}).
    \item \textbf{Playwright} -- oferuje nie tylko samo uruchamianie testów end-to-end w przeglądarce, ale również wbudowany test runner, raportowanie czy możliwość kodowania w JavaScript, Pythonie i C#.
\end{itemize}

\subsection*{Kryteria wyboru}
Decydując się na konkretny framework testowy, warto wziąć pod uwagę następujące czynniki:
\begin{itemize}
    \item \textbf{Język programowania w projekcie} -- wybór frameworku powinien być naturalnie zgodny z dominującą technologią w zespole.
    \item \textbf{Typ aplikacji i zakres testów} -- np. do testów front-endu React popularne jest Jest + React Testing Library, a do testów integracji back-endowych może się lepiej sprawdzić Pytest lub JUnit.
    \item \textbf{Możliwość integracji z CI/CD} -- jeżeli projekt wymaga ciągłej integracji, lepiej wybrać framework, który ma dobre wsparcie w popularnych narzędziach (np. GitHub Actions, Jenkins, GitLab CI).
    \item \textbf{Społeczność i dokumentacja} -- dojrzałe narzędzia z dużą społecznością pozwalają szybciej rozwiązywać problemy i uzyskać wsparcie.
\end{itemize}

% --- tu można opcjonalnie wstawić rysunek z jakimś schematem porównania frameworków, jeśli chcesz ---
% \begin{figure}[ht]
%     \centering
%     % \includegraphics[width=0.8\textwidth]{comparison.png}
%     \caption{Przykładowy schemat porównania popularnych frameworków testowych.}
%     \label{fig:framework-comparison}
% \end{figure}

\section{Biblioteki wspomagające generowanie testów}
\label{sec:biblioteki-generujace}
Współczesne narzędzia do automatycznego testowania aplikacji webowych coraz częściej oferują funkcjonalności pozwalające na częściowe lub całkowite \textit{generowanie} testów. W dużej mierze polega to na rejestrowaniu czynności użytkownika w przeglądarce (tzw. podejście \emph{record-and-play}) bądź analizie kodu źródłowego i~struktur aplikacji w celu wygenerowania przykładowych skryptów. Rozwiązania te mogą znacząco przyspieszyć proces pisania testów, zwłaszcza dla początkujących zespołów lub projektów o ograniczonych zasobach.

\subsection*{Podejście \emph{record-and-play}}
Najbardziej intuicyjnym sposobem generowania testów jest nagrywanie sekwencji działań podejmowanych przez użytkownika w~przeglądarce:
\begin{itemize}
    \item \textbf{Selenium IDE} – wtyczka do przeglądarek oparta na środowisku Selenium, umożliwia „nagrywanie” akcji, a~następnie eksport wygenerowanych skryptów w~formie testów w różnych językach (np. Java, Python, C\#). Rozwiązanie to pozwala szybko stworzyć prototypowe testy E2E, choć często wymagają one późniejszej korekty i refaktoryzacji.
    \item \textbf{Playwright Codegen} – część zestawu narzędzi \emph{Playwright}. Pozwala uruchomić przeglądarkę w~trybie interaktywnym i~rejestrować wszystkie kliknięcia, wpisywane teksty czy nawigację. Na tej podstawie generuje gotowy kod testu (np. w~TypeScripcie, Pythonie czy C\#). Wyjściowy skrypt można następnie dostosować do bardziej skomplikowanych scenariuszy.
    \item \textbf{TestCafe Recorder} (w~formie zewnętrznych wtyczek) – niektóre narzędzia dla TestCafe wspierają nagrywanie testów, choć zwykle w~mniejszym zakresie niż Selenium IDE czy Playwright.
\end{itemize}
Zaletą podejścia \emph{record-and-play} jest niski próg wejścia – wystarczy zainstalować wtyczkę lub uruchomić odpowiednie polecenie, by w~kilka minut otrzymać wstępny zestaw testów. Wadą jest jednak często mała elastyczność i~tendencja do generowania „sztywnych” skryptów, które mogą łatwo ulec awarii przy zmianach w~interfejsie użytkownika.

\subsection*{Analiza kodu i generowanie szkieletów testów}
Innym trendem jest narzędziowe wspomaganie pisania testów na bazie analizy kodu aplikacji:
\begin{itemize}
    \item \textbf{Narzędzia do analizy statycznej} – potrafią wyznaczać ścieżki wykonania w~kodzie oraz punkty potencjalnych błędów. Nie generują one pełnych testów, ale często dostarczają struktury (np. w~formie szkieletu \emph{test suite}) z uwzględnieniem klas, metod czy modułów wymagających szczególnej uwagi.
    \item \textbf{Boty testujące interfejs} – istnieją również inicjatywy open-source (np. \emph{Heuristics-based Exploratory Bots}) próbujące „klikać” w~elementy na stronie według pewnych heurystyk i w~ten sposób automatycznie odkrywać ścieżki testowe. Następnie generowane są raporty lub nawet wstępny kod testów E2E.
\end{itemize}
Rozwiązania oparte na analizie kodu lub heurystykach są bardziej zaawansowane koncepcyjnie, jednak wciąż często wymagają finalnego dostosowania testów przez doświadczonego programistę lub testera.

\subsection*{Zalety i wyzwania stosowania generatorów testów}
\textbf{Główne korzyści} z wykorzystania bibliotek wspierających generowanie testów:
\begin{itemize}
    \item \emph{Oszczędność czasu} – szybkie uzyskanie podstawowej wersji skryptów testowych.
    \item \emph{Niższy próg wejścia} – osoby początkujące w automatyzacji mogą efektywniej rozpocząć prace nad testami.
    \item \emph{Standaryzacja} – automatycznie wygenerowane testy często stosują zunifikowane wzorce i konwencje, co ułatwia późniejszą konserwację.
\end{itemize}
\textbf{Podstawowe wyzwania}:
\begin{itemize}
    \item \emph{Brak elastyczności} – wygenerowane testy bywają nadmiernie zależne od konkretnego układu strony czy struktury kodu.
    \item \emph{Konieczność refaktoryzacji} – zazwyczaj wymagane jest przeorganizowanie kodu, aby testy były utrzymywalne w dłuższym okresie.
    \item \emph{Aktualizacje przy zmianach w aplikacji} – przy każdej modyfikacji interfejsu czy logiki biznesowej część wygenerowanego kodu może tracić ważność.
\end{itemize}

\section{Narzędzia CI/CD}
\label{sec:narzedzia-ci-cd}
W projektach o~większej skali testy automatyczne nie kończą się wyłącznie na lokalnym uruchomieniu. Kluczowe znaczenie ma zintegrowanie ich z~procesem ciągłej integracji i~dostarczania (CI/CD), tak aby każdy fragment nowego kodu był weryfikowany pod kątem poprawności i~stabilności. Poniżej przedstawiono najpopularniejsze narzędzia CI/CD oraz sposoby włączania do nich testów automatycznych.

\subsection*{GitLab CI}
Platforma GitLab oferuje wbudowany system CI/CD oparty na pliku \emph{.gitlab-ci.yml}. Aby dołączyć testy webowe (np. napisane z~użyciem Playwright, Selenium czy Cypress), należy:
\begin{itemize}
    \item Zdefiniować etap (job) w sekcji \emph{test}, w którym instalowane są zależności (np. \emph{npm install} w przypadku JavaScript/TypeScript),
    \item Uruchomić polecenie wywołujące testy (np. \emph{npx playwright test}),
    \item Skonfigurować artefakty, czyli np. raport HTML lub zrzuty ekranu w razie niepowodzenia testów.
\end{itemize}
Przykład fragmentu \emph{.gitlab-ci.yml} może wyglądać następująco:
\begin{verbatim}
stages:
  - build
  - test

e2e_tests:
  stage: test
  image: mcr.microsoft.com/playwright:focal
  script:
    - npm ci
    - npx playwright test
  artifacts:
    when: on_failure
    paths:
      - playwright-report
\end{verbatim}

\subsection*{GitHub Actions}
\emph{GitHub Actions} to kolejne popularne narzędzie do automatyzacji zadań w cyklu rozwoju oprogramowania. Plik konfiguracyjny (np.~\emph{.github/workflows/test.yml}) definiuje \emph{jobs}, które można uruchamiać przy zdarzeniach takich jak \emph{push} czy \emph{pull request}. W kontekście testów automatycznych częstą praktyką jest:
\begin{itemize}
    \item Wybór obrazu bazowego (np. \emph{ubuntu-latest}),
    \item Instalacja wymaganych bibliotek i narzędzi (Selenium, Playwright),
    \item Uruchomienie skryptów testowych,
    \item Publikacja wyników testów jako część \emph{Actions artifacts}.
\end{itemize}

\subsection*{Jenkins}
\emph{Jenkins} to jedno z najstarszych i najbardziej konfigurowalnych narzędzi CI/CD typu open source. Dla testów webowych zazwyczaj przygotowuje się \emph{pipeline}, w którym:
\begin{itemize}
    \item Następuje pobranie kodu z repozytorium,
    \item Uruchamiane są komendy instalacji i testów,
    \item Generowany jest raport (np. w formacie JUnit, HTML czy Allure),
    \item Raport prezentowany jest w~interfejsie Jenkins po zakończeniu procesu.
\end{itemize}
Dzięki bogatej kolekcji wtyczek (\emph{plugins}) Jenkins może integrować się z wieloma innymi usługami i narzędziami, m.in. z Selenium Grid, Dockerem czy Kubernetesem, ułatwiając równoległe uruchamianie testów na różnych konfiguracjach.

\subsection*{Kluczowe korzyści integracji testów z CI/CD}
\begin{itemize}
    \item \textbf{Bieżąca kontrola jakości} – przy każdym \emph{commit} testy są wywoływane automatycznie, co umożliwia szybką reakcję na regresje.
    \item \textbf{Automatyzacja raportowania} – wyniki testów są od razu dostępne dla zespołu, w tym graficzne raporty, zrzuty ekranu czy logi.
    \item \textbf{Możliwość szybkiej dystrybucji} – w przypadku pozytywnego przejścia testów, aplikacja może być automatycznie wdrożona (ang. \emph{continuous delivery}) na docelowe środowisko.
    \item \textbf{Elastyczność} – narzędzia CI/CD pozwalają na uruchamianie testów w różnych wariantach, konfiguracjach przeglądarek czy systemach operacyjnych.
\end{itemize}

\subsection*{Wyzwania i dalsze możliwości rozwoju}
\begin{itemize}
    \item \textbf{Konfiguracja środowisk} – niekiedy wymagana jest prekonfiguracja usług (np. bazy danych, mikroserwisów), co zwiększa poziom złożoności.
    \item \textbf{Czas wykonania testów} – duże projekty posiadające setki testów E2E mogą generować długi czas budowy. Rozwiązaniem jest równoległe uruchamianie i skalowanie, np. na klastrze Kubernetes.
    \item \textbf{Utrzymanie testów} – wraz z rozwojem aplikacji rośnie liczba testów i ich poziom skomplikowania. Testy muszą być regularnie aktualizowane, aby nie blokowały procesu wdrożenia fałszywymi alarmami.
\end{itemize}

\bigskip
\noindent
\textbf{Podsumowując}, włączenie testów automatycznych do procesu CI/CD stanowi nieodzowny element profesjonalnego cyklu wytwarzania oprogramowania, zapewniając ciągłą weryfikację jakości i szybką detekcję błędów. Integracja z narzędziami typu GitLab CI, GitHub Actions czy Jenkins umożliwia standaryzację procesu testowania, a także automatyzację raportowania wyników i uproszczoną ścieżkę wdrożenia na środowiska testowe oraz produkcyjne.

\chapter{Strategie generacji testów}
{Przedstaw wybrane podejścia do automatycznego lub półautomatycznego generowania testów.}

\section{Generowanie testów na podstawie rejestrowania akcji}
{Opis: rejestrowanie czynności użytkownika w przeglądarce (np. za pomocą \emph{Selenium IDE}, \emph{Playwright Codegen}), a następnie generowanie skryptów testowych.}

\section{Generowanie testów na podstawie modeli}
{Możesz omówić krótko podejście \emph{Model-Based Testing}, gdzie testy są generowane z modelu zachowań systemu, stanów i przejść.}

\section{Analiza statyczna i tworzenie testów}
{Wspomnij, że niektóre narzędzia potrafią analizować kod aplikacji i np. na podstawie ścieżek wywołań generować scenariusze testowe.}

\chapter{Architektura oraz interfejs aplikacji}
{W tym rozdziale opiszesz konstrukcję swojego narzędzia, jego moduły, sposób komunikacji itp.}

\section{Ogólny zarys architektury}
{Zwięźle opisz główne moduły aplikacji (np. moduł generowania testów, moduł interfejsu graficznego/CLI, moduł zarządzania danymi). Wskaż, w jaki sposób te moduły się ze sobą komunikują.}

\section{Interfejs użytkownika}
{Zaprezentuj, jak wygląda interfejs (graficzny lub konsolowy) Twojej aplikacji:
\begin{itemize}
    \item główne ekrany/formularze,
    \item sposób nawigacji,
    \item kluczowe funkcje (np. generowanie pliku testowego, zapis ustawień).
\end{itemize}
Możesz dołączyć zrzuty ekranu i omówić je.}

\chapter{Omówienie kodu źródłowego aplikacji}
{Przedstaw strukturę plików oraz folderów oraz wyjaśnij najważniejsze fragmenty implementacji. Wskaż, które biblioteki z rozdziału o „Dostępnych technologiach” zostały wykorzystane i w jaki sposób.}

\section{Struktura projektu}
{Opisz, w jaki sposób podzieliłeś/łaś projekt na moduły, pakiety, foldery:
\begin{itemize}
    \item logika generowania testów,
    \item definicja modelu danych (np. dane o testach, konfiguracje),
    \item klasy pomagające w tworzeniu raportów lub integracji z CI.
\end{itemize}}

\section{Kluczowe fragmenty kodu}
{Zademonstruj przykłady najważniejszych funkcji/metod, np.:
\begin{itemize}
    \item kod odpowiedzialny za rejestrowanie interakcji,
    \item generowanie plików testowych,
    \item konwersję danych do konkretnych frameworków.
\end{itemize}
Przedstaw je w formie listingów z krótkim komentarzem.}

\chapter{Przykłady użyć aplikacji}
{W tym rozdziale możesz zaprezentować, jak Twoje narzędzie działa w praktyce.}

\section{Uruchamianie narzędzia w trybie interaktywnym}
{Pokaż krok po kroku, jak użytkownik wypełnia formularz czy wybiera opcje w CLI, aby wygenerować testy. Zademonstruj końcowe pliki testowe.}

\section{Integracja z procesem CI/CD}
{Wyjaśnij, jak można włączyć wygenerowane testy do automatycznego pipeline’u, np. w GitLab CI, Jenkinsie, itp. Przedstaw przykładowy fragment konfiguracji.}

\chapter{Podsumowanie}
{Na koniec podsumuj swoją pracę, oceń jej efekty i zaproponuj kierunki rozwoju.}

\section{Wnioski końcowe}
{Opisz, w jakim stopniu udało się osiągnąć założone cele, co było największym wyzwaniem, co dało najwięcej satysfakcji/rezultatów.}

\section{Możliwości dalszego rozwoju}
{Opisz, jakie jeszcze funkcjonalności można by dodać do narzędzia, jakie usprawnienia byłyby przydatne, jakie biblioteki lub technologie mogłyby zostać wykorzystane w przyszłości.}

\begin{thebibliography}{9}
    \bibitem{roman2024} Roman, A., \& Zmitrow, K. (2024). \textit{Testowanie oprogramowania w praktyce: studium przypadków 2.0}.
    \bibitem{osherove2024} Osherove, R. (2024). \textit{Testy jednostkowe: świat niezawodnych aplikacji}.
    \bibitem{roman2024_case} Roman, A., \& Zmitrow, K. (2024). \textit{Testowanie oprogramowania w praktyce: studium przypadków}.
    \bibitem{roman2024_quality} Roman, A. (2024). \textit{Testowanie i jakość oprogramowania: modele, techniki, narzędzia}.
    \bibitem{circleci} CircleCI. (n.d.). What is End-to-End Testing? Pozyskano z \url{https://circleci.com/blog/what-is-end-to-end-testing/}
    \bibitem{playwright} Microsoft Playwright. (n.d.). Introduction to Playwright. Pozyskano z \url{https://playwright.dev/docs/intro}
\end{thebibliography}

\end{document}
