
\documentclass[12pt]{report}
\usepackage[T1]{fontenc} % Obsługa polskich znaków
\usepackage[utf8]{inputenc}
\usepackage[polish]{babel}
\usepackage{geometry}
\geometry{a4paper, margin=1in}
\usepackage{tocbibind} % Dodaje spis treści do spisu treści
\usepackage{hyperref} % Pakiet do obsługi linków

\title{Narzędzie wspierające tworzenie testów dla aplikacji webowych}
\author{
    Łukasz Kowalewski \\ % Autor pracy
    \vspace{0.5cm} % Odstęp między autorem a promotorem
    Promotor: dr inż. Marcin Adamski % Promotor
}
\date{}

\begin{document}

\maketitle

\tableofcontents
\newpage

\chapter{Wstęp}

W niniejszym wstępie postaram się uzasadnić istotność tematu pracy inżynierskiej. Tytuł pracy "Narzędzie wspierające tworzenie testów dla aplikacji webowych" opisuje funkcję jaką stworzony w ramach pracy inżynierskiej program ma za zadanie spełnić. Jego rolą jest dostarczenie użytkownikowi narzędzia, które pozwoliłoby mu dostarczać oprogramowania oraz testów lepszej jakości w krótszym czasie. Następne rozdziały postarają się omówić problematykę zagadnienia oraz możliwe rozwiązania. By zrozumieć i tym samym znaleźć motywację by owe zagadnienia analizować, należałoby uzasadnić potrzebę tworzenia takiego rodzaju oprogramowania.

Testy stanowią kluczowy element każdego zaawansowanego oraz dojrzałego systemu informatycznego bądź aplikacji. Jest to fakt uznany i łatwy do wykazania zwracając uwagę choćby na istnienie specjalnych oddziałów w korporacjach zajmujących się wyłącznie tworzeniem testów, bądź zauważając wysoki popyt na programistów wyspecjalizowanych w tej dziedzinie. To właśnie takie korporacje najczęściej tworzą zaawansowane narzędzia typu omawianego w powyższej pracy inżynierskiej. W tak dużych projektach wszelkie błędy systemu potrafią odbić się na bardzo wysokich stratach finansowych, one właśnie stanowią główne źródło motywacji by w ogóle te testy tworzyć. Jednocześnie warto zwrócić uwagę na wprost przeciwny obraz sytuacji dla projektów średnich bądź małych rozmiarów. W tego typu przedsięwzięciach często brakuje źródła motywacji m. in. w postaci potencjalnych wysokich strat finansowych, co skutkuje sytuacją w której element testów jest pomijany z wzglądu na wysoki koszt czasowy ich tworzenia.

Głównym celem omawianej aplikacji będzie wyjście naprzeciw potrzebom ułatwienia tworzenia w krótkim czasie dużej ilości prostych testów tak by małe i średnie projekty możliwie niskim kosztem uzyskały wysokie pokrycie testowe.

\chapter{Przykłady zastosowań testów}
% Treść rozdziału 2

\chapter{Dostępne technologie}
% Treść rozdziału 3

\chapter{Strategie generacji testów}
% Treść rozdziału 4

\chapter{Architektura oraz interfejs aplikacji}
% Treść rozdziału 5

\chapter{Omówienie kodu źródłowego aplikacji}
% Treść rozdziału 6

\chapter{Przykłady użyć aplikacji}
% Treść rozdziału 7

\chapter{Podsumowanie}
% Treść rozdziału 8

\begin{thebibliography}{9}
    \bibitem{roman2024} Roman, A., \& Zmitrow, K. (2024). \textit{Testowanie oprogramowania w praktyce: studium przypadków 2.0}.

    \bibitem{osherove2024} Osherove, R. (2024). \textit{Testy jednostkowe: świat niezawodnych aplikacji}.

    \bibitem{roman2024_case} Roman, A., \& Zmitrow, K. (2024). \textit{Testowanie oprogramowania w praktyce: studium przypadków}.

    \bibitem{roman2024_quality} Roman, A. (2024). \textit{Testowanie i jakość oprogramowania: modele, techniki, narzędzia}.

    \bibitem{circleci} CircleCI. (n.d.). What is End-to-End Testing? Pozyskano z \url{https://circleci.com/blog/what-is-end-to-end-testing/}

    \bibitem{playwright} Microsoft Playwright. (n.d.). Introduction to Playwright. Pozyskano z \url{https://playwright.dev/docs/intro}
\end{thebibliography}

\end{document}